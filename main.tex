\documentclass{article}
\usepackage[utf8]{inputenc} % bibliography
\usepackage[square,sort,comma,numbers]{natbib}
\bibliographystyle{agsm}
\bibpunct{(}{)}{;}{a}{,}{,}
\usepackage{graphicx, adjustbox, rotating} % adding immages
\usepackage{booktabs, url}
\usepackage{comment}
\usepackage{float}
\usepackage{pdflscape}
\usepackage{caption, subcaption}
\usepackage[margin=1in]{geometry} %margins
\usepackage[colorlinks = true, urlcolor = blue, linkcolor = blue, citecolor = blue]{hyperref}
%\usepackage[parfill]{parskip} % first line skip 
\setlength{\parindent}{15pt} %for paragraph indent
\usepackage{indentfirst} %for first paragraph indent
\usepackage{setspace} %line spacing
 \usepackage{pdfpages}
\usepackage{tikz}
\def\checkmark{\tikz\fill[scale=0.4](0,.35) -- (.25,0) -- (1,.7) -- (.25,.15) -- cycle;} 
\DeclareGraphicsExtensions{.pdf,.png,.jpg}
\pdfgentounicode=1


\begin{comment}
(1) succinctly summarize the issues, 
(2) outline how you address them empirically (e.g. data, empirical design, and basic models), 
(3) discuss your empirical results, and 
(4) conclude with what it all means.
\end{comment}


\title{Problem Set 2: The Effects of Medicaid Expansions on Childbearing Decisions}
\author{Mena Kiser and Ariana Rodriguez}
\date{\today}
\doublespacing

\begin{document}

\maketitle

%\begin{abstract}
 %   In 2014, the Affordable Care Act (ACA) expanded Medicaid to cover to all low-income adults as opposed to primarily families with children. This study exploits the state differential take-up of this policy to evaluate effects on childbearing decisions in the targeted population. Using 2008-2019 American Community Survey (ACS), we implement a XX design with treatment defined as XX
%\end{abstract}

\onehalfspacing
% I think we can remove the abstract?
\section{Introduction}
In 2014, the Affordable Care Act (ACA) expanded Medicaid to cover all low–income adults, opposed to primarily covering families with children. Previous research on this expansion has found increases in healthcare coverage and reductions in mortality, especially among younger and less-educated populations. Our research further investigates the impact of this expansion on childbearing among these groups. Given the broader coverage, the expansion may reduce the incentive to have children as a way to obtain Medicaid coverage, while also reducing the opportunity cost of childbearing through greater financial and health security. The net effect on fertility therefore requires empirical investigation to understand the direction of impacts on fertility. 

We exploit the staggered adoption of the Medicaid expansion across states to estimate its impact on insurance uptake, employment, and childbearing among young, less-educated adults. By focusing on a group most affected by the policy, low-income adults aged 26–-30 without a high school degree, we assess whether the ACA expansion not only improved health coverage but also influenced broader family decisions. Our findings confirm substantial gains in insurance coverage following the expansion, with Medicaid coverage increasing by roughly 11 percentage points and overall insurance coverage by 10 percentage points. In contrast, we find no significant changes in employment or marital status, suggesting that expanded access to Medicaid did not discourage work or alter marriage incentives among this group.

Turning to fertility outcomes, we find that exposure to the Medicaid expansion is associated with a modest reduction in the number of children per adult and a lower probability of having a new baby. Specifically, the expansion is associated with a decrease of about 0.04 children per household and a 1.4 percentage point decline in the likelihood of having a child aged one or younger. These effects are somewhat stronger after controlling for Medicaid coverage, implying that the expansion’s influence extends beyond direct enrollment, potentially through (unobserved) household spillovers or improved access to family planning and reproductive health services. Heterogeneity analyses indicate that these effects are concentrated among men, non-employed, and Hispanic adults, groups more likely to gain coverage under the expansion, while employed and non-Hispanic adults show little change in fertility behavior.

Overall, our results suggest that the ACA Medicaid expansion modestly slowed or delayed fertility among low-educated young adults, likely reflecting changes in both economic security and access to health care rather than shifts in labor or marital status. 


\section{Data}
Our data comes from the 2008-2019 American Community Survey (ACS), obtained through the Integrated Public Use Microdata Series (IPUMS). The ACS is a yearly cross-sectional, one percent, annual survey of households in the United States collecting detailed information on social, economic, housing, and demographic characteristics of the U.S. population. The survey records respondent's marital status, number of children (and the age of their youngest child), type of insurance coverage, and employment behavior. We note that `children' is defined as own children in the household, including step-children, and excluding biological and step-children living elsewhere.

Our focus sample includes years 2008-2019, all 50 U.S. states and D.C., restricting to respondents of ages 26--30 with an educational level below a high school degree (or equivalent). We observe $173,554$ adults, and we assume each observation represents a distinct household. During this time 34 states enacted the ACA Medicaid expansion, 3 enacted it after 2019, and 14 other states never enacted it. For outcomes, we define having a new baby as individuals whose youngest child is 1 year old or younger. Key outcomes include the number of children, and whether an individual has a new baby (defined as the youngest child being 1 year old or younger), as well as employment indicators such as being employed or a full-time worker (working at least 40 hours per week). For insurance coverage types, we have binary indicators for whether the respondent reported having any insurance, Medicaid coverage, public insurance coverage, or coverage through their employer.

Table \ref{tab:sumstat} presents summary statistics of the main variables. Around 58\% of the sample is male, 61\% is white, and 61\% are single (divorced, widowed, or never married). Basic demographics, including gender, age, and racial composition, are similar across never-treated states and those that eventually expanded Medicaid. However, among treated states, the post-expansion period is associated with a lower share of married adults (14 percent decrease), fewer new parents (16 percent decrease), and slightly fewer children per household (11 percent decrease). These descriptive differences suggest that family changes may have slowed down following the expansion, though we formally test this below after accounting for state and year fixed effects \footnote{See Table \ref{tab:cov}, Panel B, Column 1, which shows a statistically insignificant $-0.009$ percentage point change in marriage rates post-expansion.}.

Regarding employment outcomes, we can see that employment, usual hours worked, and share of full time workers remains within a close range (within $2\%$), while earned income increases for those in the post period by $17\%$ or $\$$2,229 and health insurance coverage through an employer increases by $19\%$. As for the marital status, we should test whether these differences are robust to state and year fixed effects \footnote{We test this in Panel B, Columns 2-4 of Table \ref{tab:cov} and find small, statistically insignificant changes in employment, earned income, and full time employment status.}.

Finally, we see clear differences in health coverage patterns. Individuals in treated states are more likely to have any health insurance coverage (23\% higher) or Medicaid coverage (55\%) than those in non–expanding states. Treated states in the post period additionally increase any coverage by 46\% and Medicaid coverage by 64\%. The patterns suggest there were substantial gains in coverage following the expansion, consistent with the ACA's intended effects.

\section{Empirical Strategy and Results}
To identify whether the Medicaid expansion translated into higher Medicaid coverage take up rates and evaluate its effect in childbearing decisions, we implement a difference-in-difference approach through the following regression:

\begin{equation}
    \label{eq:did}
    Y_{ist} = \beta_0 + \beta_1 expansion_{s}*post_{st} + X_{ist}\gamma + Z_{s} + V_{y}+ + \varepsilon_{ist}
\end{equation}

where $Y_{ist}$ is the outcome of interest for individual $i$ in state $s$ and year $t$; ${expansion}_s$ is an indicator for states that ever adopted the Medicaid expansion; and ${post}_{st}$ is an indicator equal to one in years in which the expansion is in effect in state $s$. The coefficient of interest, $\beta_1$, captures the average effect of being exposed to the Medicaid expansion on the outcome.

The vector $X_{ist}$ includes individual-level controls for sex, age, race, Hispanic origin, marital status, education, and employment behavior, which includes indicators for being employed, full time work status (work week of at least 40 hours), earned income, and usual weekly hours worked. We also include state fixed effects $Z_{s}$ to account for time-invariant differences across states and year fixed effects $V_{y}$ to capture common shocks affecting all states in a given year. Because treatment varies at the state–year level, we cluster standard errors at the state level to allow for correlation within states over time.

Panel A of Table \ref{tab:cov}, presents estimates for Equation \ref{eq:did}, with outcomes being different health insurance coverage types. We see that after the Medicaid expansion, people in treated states see an increase in any health insurance coverage by $10.7$ percentage points (p.p) (Column 1) while Medicaid coverage increased by $10.8$ p.p (Column 2). Public insurance increases by $11.1$ p.p (Column 3), which signals the increase comes largely from the Medicaid increase. These magnitudes imply roughly 25–45 percent gains relative to pre-expansion means, confirming that the ACA expansion substantially raised coverage among our population. We see no significant changes in coverage through employer (Column 4). Furthermore, we see differences in effects by sex, Hispanic origin, and employment status, thus we choose these parameters to later evaluate heterogeneity of results for the effects of the expansion on childbearing. 

% is this too tangled?
In Panel A of Table \ref{tab:cov}, we can also see that the effect of being currently married and the probability of being employed are positively associated with having any type of health insurance coverage and coverage through an employer, and negatively correlated with having Medicaid coverage. This can be explained by how by being married you are more likely to have an employed adult in your household and receive coverage through their employer. We explore if the Medicaid expansion made people less likely to be married or employed as receiving Medicaid coverage may reduce incentives for getting married or being employed and obtaining health insurance coverage through their employer. Panel B examines whether the expansion affected marriage or employment behavior. The dependent variables are binary indicators for being currently married, employed, working full time, and continuous measures of earned income (in current USD) and usual weekly hours worked. The expansion does not lead to statistically significant changes in marriage rates ($-0.9$ p.p.). We can associate the Medicaid expansion with a negligible increase in employment by $0.3$ p.p, a decrease in full time status by $-.01$ p.p, an increase in earned income of $141$ USD, and an increase in usual weekly hours worked by $0.05$ hours. 

Overall, these results suggest that access to Medicaid coverage did not discourage labor among low-income young adults, contrary to concerns that expanded public insurance might reduce work incentives. We therefore treat employment behavior and marital status as potential controls rather than outcomes of interest in subsequent analyses of fertility and childbearing.

Next, we evaluate whether the Medicaid expansion influenced childbearing decisions. Panel A of Table \ref{tab:child_did} uses the number of children as the dependent variable, capturing changes in the stock of children. Panel B uses an indicator for having a new baby (defined as having a child aged one year or younger) to measure changes in the flow of births.

Our preferred specification is Column 3, which includes all aforementioned controls and employment and marriage controls. Exposure to the Medicaid expansion is associated with a decrease of 0.042 children per adult, significant at the 1 percent level. While the magnitude is small, it represents roughly a 4 percent reduction relative to the pre-expansion mean and suggests that the expansion may have modestly reduced the pace of family growth among low-educated young adults. Consistent with this pattern, in Panel B we find that the probability of having a new baby declines by 1.4 percentage points, significant at the 10 percent level. 

After controlling for having Medicaid coverage, we see that the Medicaid expansion is associated with a higher decrease in the number of children of 0.101 average children, significant at the 1 percent level. This suggests that the effect of the Medicaid expansion extends beyond Medicaid coverage. This could be due to your partner being covered or benefiting indirectly from expanded eligibility, changes in access to family planning or prenatal care, or broader behavioral responses to the expansion—such as improved financial security or labor market attachment—that influence fertility decisions.

The direction of these effects aligns with the idea that broader access to health insurance can alter fertility decisions. Prior to the ACA, individuals without children often faced limited access to Medicaid; by removing this dependency, the expansion may have reduced the incentive to have children as a means to qualify for coverage. At the same time, reduced financial and health-related uncertainty could delay or space out childbearing decisions by increasing access to birth control. To examine the timing and dynamics of the observed effects on childbearing, we estimate an event-study specification that extends the difference-in-differences framework. Specifically, we estimate:

\begin{equation}
\label{eq:event}
Y_{ist} = \sum_{k=-5}^{5} \beta_k D_{s,t+k} + X_{ist}\gamma + Z_s + V_t + \varepsilon_{ist},
\end{equation}

Where $D_{s,t+k}$ is a set of relative year indicators that capture the number of years before or after a state $s$ implemented the expansion. The omitted category is the year immediately before the expansion ($k=-1$). All regressions include the same set of controls and fixed effects as out difference in difference, with standard errors clustered at the state level. This specification further allows us to test pre–trends for our treated and never treated states. We include up to five leads and lags of treatment exposure, and weight observations using person-level survey weights. Our event study specification focuses on childbearing outcomes as differential impacts by subgroups. 

Figure \ref{fig:es_nchild} plots the effects of the Medicaid expansion on the number of children in the household for up to five years before and after states adopted the policy. Panel (a) shows that overall, pre–treatment coefficients are statistically insignificant, supporting the parallel trends assumption. Beginning the year of expansion and for the years afterward, point estimates are consistently negative, suggesting a modest decline in the number of children induced by the Medicare expansion. Although imprecise, the persistence of negative coefficients aligns with the difference-in-differences results in Table \ref{tab:child_did}, where exposure reduced the average number of children by roughly 0.04.

The heterogeneity panels (b)--(g) show that the decline are primarily driven by men, those not employed, and Hispanic individuals. This pattern could suggest that reductions in household child counts were not universal but concentrated among subpopulations more likely to rely on public insurance coverage. Among those already participating in the labor market or with alternative access to employer-provided insurance, the expansion may have had little effect on family size. However, none of these estimates are statistically different from zero.

Figure \ref{fig:es_newbaby}, panel (a) examines the probability of a having a new baby (a child aged 1 or less). Although pre–trends are not as strong as in the previous figure, there does not appear to be a systematic difference in pre–trends. Post-expansion, the estimates turn modestly negative, with some positive estimates in years 3 and 4. These dynamics suggest that fertility responses occurred gradually as adults adjusted to the new coverage environment rather than immediately upon policy adoption.

Heterogeneity patterns in Figure \ref{fig:es_newbaby} are mixed rather than consistently negative. The point estimates for males and employed adults tend to be slightly positive after the expansion, although the confidence intervals are wide and include zero, except for year 3. This suggests that for these groups may already have stable income or access to employer coverage, and the expansion did not reduce the likelihood of new births and may even have loosened financial constraints associated with childbearing. By contrast, non-employed, Hispanic and female respondents show mostly small negative effects, consistent with a modest reduction or delay in fertility among those for whom Medicaid provided a new source of insurance security.

\section{Remarks}
The findings from both the difference-in-difference and event study analyses indicate that the ACA Medicaid expansion had modest effects on childbearing behavior among low-income and low education adults. The expansion significantly increased Medicaid and overall insurance coverage, confirming that the policy reached its intended population. However, it did not meaningfully impact employment outcomes or marital status, suggesting that any changes to fertility rates are unlikely to be driven by these channels. Across specifications, exposure to the expansion is associated with small reductions in the number of children and lower probability of having a new baby. Event study estimates confirm these patterns, although through imprecise estimates. Heterogeneity analyses suggest that declines in fertility are concentrated among Hispanic and non-employed adults, who might be more likely to gain coverage through Medicaid. Together, these results suggest that expanded public insurance may have modestly reduced or delayed fertility among groups that may have previously relied on children to gain health insurance coverage. For others, like the employed population, the expansion seems to have little effect on fertility decisions. These findings contribute to a broader understanding of how health policy can shape household demographics. Large increases in medical care do not necessarily lead to large shocks to fertility or labor market outcomes. Future research could examine the role of partner coverage, contraceptive access, or state-level policy environments to better understand the pathways linking public insurance expansions and fertility decisions.

\newpage
\section{Tables}

\begin{table}[h!]
\begin{centering}
\singlespacing
\caption{Summary Statistics}
\adjustbox{width=\linewidth, center}{
    \begin{tabular}{lcccccccc}
\toprule
\toprule
  & & & & & \multicolumn{4}{c}{State Ever Treated} \\
 Variable & \multicolumn{2}{c}{All} & \multicolumn{2}{c}{Never Treated} & \multicolumn{2}{c}{Pre Expansion} & \multicolumn{2}{c}{Post Expansion} \\
\midrule 
 Male   & 0.584 & (0.002)  & 0.591 & (0.002)  & 0.578 & (0.002)  & 0.580 & (0.003)  \\
 Age   & 28.077 & (0.004)  & 28.080 & (0.007)  & 28.058 & (0.007)  & 28.100 & (0.009)  \\
 Race   & & & & & \\
 \hspace{0.3cm}  White   & 0.606 & (0.002)  & 0.646 & (0.002)  & 0.594 & (0.002)  & 0.562 & (0.003)  \\
 \hspace{0.3cm}  Black   & 0.145 & (0.001)  & 0.176 & (0.002)  & 0.123 & (0.002)  & 0.131 & (0.002)  \\
 \hspace{0.3cm}  Native American   & 0.015 & (0.000)  & 0.012 & (0.000)  & 0.015 & (0.001)  & 0.018 & (0.001)  \\
 \hspace{0.3cm}  Asian   & 0.027 & (0.000)  & 0.014 & (0.001)  & 0.030 & (0.001)  & 0.043 & (0.001)  \\
 \hspace{0.3cm}  Other   & 0.208 & (0.001)  & 0.152 & (0.002)  & 0.239 & (0.002)  & 0.246 & (0.003)  \\
 Hispanic origin   & 0.512 & (0.002)  & 0.491 & (0.003)  & 0.537 & (0.002)  & 0.507 & (0.003)  \\
 Marital status   & & & & & \\
 \hspace{0.3cm} Currently married   & 0.361 & (0.001)  & 0.375 & (0.002)  & 0.373 & (0.002)  & 0.320 & (0.003)  \\
 \hspace{0.3cm} Separated   & 0.034 & (0.001)  & 0.042 & (0.001)  & 0.032 & (0.001)  & 0.026 & (0.001)  \\
 \hspace{0.3cm} Single   & 0.605 & (0.001)  & 0.582 & (0.002)  & 0.595 & (0.002)  & 0.654 & (0.003)  \\
 Number of children   & 1.135 & (0.004)  & 1.166 & (0.007)  & 1.170 & (0.007)  & 1.036 & (0.009)  \\
 Age of youngest child   & 3.437 & (0.014)  & 3.505 & (0.023)  & 3.351 & (0.021)  & 3.470 & (0.030)  \\
 Has baby age $<$1   & 0.165 & (0.001)  & 0.163 & (0.002)  & 0.176 & (0.002)  & 0.154 & (0.002)  \\
 New parent   & 0.034 & (0.001)  & 0.031 & (0.001)  & 0.037 & (0.001)  & 0.031 & (0.001)  \\
 Insurance coverage   & & & & & \\
 \hspace{0.3cm}  Any coverage   & 0.446 & (0.002)  & 0.348 & (0.002)  & 0.427 & (0.002)  & 0.623 & (0.003)  \\
 \hspace{0.3cm}  Coverage through employer   & 0.183 & (0.001)  & 0.170 & (0.002)  & 0.178 & (0.002)  & 0.211 & (0.003)  \\
 \hspace{0.3cm}  Public insurance coverage   & 0.248 & (0.001)  & 0.160 & (0.002)  & 0.241 & (0.002)  & 0.394 & (0.003)  \\
 \hspace{0.3cm}  Coverage through Medicaid   & 0.239 & (0.001)  & 0.151 & (0.002)  & 0.234 & (0.002)  & 0.383 & (0.003)  \\
 Educational attainment   & & & & & \\
 \hspace{0.3cm} Grade $<=$4   & 0.042 & (0.001)  & 0.041 & (0.001)  & 0.044 & (0.001)  & 0.041 & (0.001)  \\
 \hspace{0.3cm}  Grades 5--8   & 0.244 & (0.001)  & 0.235 & (0.002)  & 0.262 & (0.002)  & 0.230 & (0.003)  \\
 \hspace{0.3cm} Grade 9   & 0.176 & (0.001)  & 0.191 & (0.002)  & 0.171 & (0.002)  & 0.160 & (0.002)  \\
 \hspace{0.3cm} Grade 10   & 0.196 & (0.001)  & 0.208 & (0.002)  & 0.190 & (0.002)  & 0.185 & (0.002)  \\
 \hspace{0.3cm} Grade 11   & 0.266 & (0.001)  & 0.258 & (0.002)  & 0.268 & (0.002)  & 0.277 & (0.003)  \\
 Educational attainment of spouse   & 0.032 & (0.001)  & 0.034 & (0.001)  & 0.034 & (0.001)  & 0.026 & (0.002)  \\
 \hspace{0.3cm} Grade $<=$4   & & & & & \\
 \hspace{0.3cm}  Grades 5--8   & 0.184 & (0.002)  & 0.175 & (0.003)  & 0.198 & (0.003)  & 0.175 & (0.004)  \\
 \hspace{0.3cm} Grade 9   & 0.095 & (0.001)  & 0.100 & (0.002)  & 0.098 & (0.002)  & 0.083 & (0.003)  \\
 \hspace{0.3cm} Grade 10   & 0.064 & (0.001)  & 0.070 & (0.002)  & 0.065 & (0.002)  & 0.051 & (0.002)  \\
 \hspace{0.3cm} Grade 11   & 0.077 & (0.001)  & 0.080 & (0.002)  & 0.079 & (0.002)  & 0.069 & (0.002)  \\
 \hspace{0.3cm} Grade 12   & 0.347 & (0.002)  & 0.347 & (0.004)  & 0.342 & (0.003)  & 0.356 & (0.005)  \\
 \hspace{0.3cm} 1 year of college   & 0.096 & (0.001)  & 0.097 & (0.002)  & 0.091 & (0.002)  & 0.104 & (0.003)  \\
 \hspace{0.3cm} 2 years of college   & 0.032 & (0.001)  & 0.031 & (0.001)  & 0.031 & (0.001)  & 0.037 & (0.002)  \\
 \hspace{0.3cm} 4 years of college   & 0.033 & (0.001)  & 0.030 & (0.001)  & 0.029 & (0.001)  & 0.044 & (0.002)  \\
 \hspace{0.3cm} 5$+$ years of college   & 0.008 & (0.000)  & 0.006 & (0.001)  & 0.007 & (0.001)  & 0.012 & (0.001)  \\
 Employed   & 0.559 & (0.002)  & 0.560 & (0.002)  & 0.556 & (0.002)  & 0.562 & (0.003)  \\
 Earned income   & 13,713.512 & (57.020)  & 13,237.615 & (86.925)  & 13,119.026 & (84.799)  & 15,347.894 & (137.358)  \\
 Usual weekly hours worked   & 25.585 & (0.061)  & 25.935 & (0.101)  & 25.508 & (0.098)  & 25.166 & (0.125)  \\
 Full time (40$+$ work hrs)   & 0.456 & (0.002)  & 0.463 & (0.003)  & 0.452 & (0.002)  & 0.454 & (0.003)  \\
\\
Sample size & \multicolumn{2}{c}{173,554}  & \multicolumn{2}{c}{65,793}  & \multicolumn{2}{c}{64,529}  & \multicolumn{2}{c}{43,232}  \\
\bottomrule
\bottomrule
\end{tabular}
}
\label{tab:sumstat}
\end{centering}
\begin{spacing}{1}
\begin{footnotesize}

\textit{Notes:} This table summarizes main respondent characteristics. Marital status is grouped into currently married (with spouse present or absent), separated, or single (divorced, widowed, never married). Public insurance includes federal insurance programs of three kinds: Medicare, Medicaid, and Department of Veterans Affairs insurance.
\end{footnotesize}
\end{spacing}
\end{table}

\newpage
\begin{table}[h!]
\begin{centering}
\singlespacing
\caption{Insurance coverage}
\adjustbox{width=\linewidth, center}{
    \begin{tabular}{lcccc}
\toprule
\toprule
 \multicolumn{5}{c}{Panel A: Insurance coverage } \\
 Independent Variable & Any insurance & Medicaid & Public insurance & Coverage through \\
  & coverage & coverage & coverage &  employer \\
  & (1) & (2) & (3) &  (4) \\
\midrule 
 Medicaid expansion*Post   & 0.107***   & 0.108***   & 0.111***   & 0.001  \\
 & (0.014)   & (0.014)   & (0.014)   & (0.008)  \\
 Male   & -0.163***   & -0.130***   & -0.129***   & -0.033***  \\
 & (0.009)   & (0.012)   & (0.012)   & (0.005)  \\
 Hispanic origin   & -0.197***   & -0.110***   & -0.117***   & -0.069***  \\
 & (0.014)   & (0.013)   & (0.013)   & (0.006)  \\
 Currently married   & 0.052***   & -0.012   & -0.016**   & 0.066***  \\
 & (0.009)   & (0.007)   & (0.007)   & (0.006)  \\
 Employed   & 0.051***   & -0.075***   & -0.079***   & 0.136***  \\
 & (0.013)   & (0.011)   & (0.011)   & (0.011)  \\
\\
Other employment covariates  & $\checkmark$ & $\checkmark$ & $\checkmark$ & $\checkmark$ \\
R-2 & 0.153 & 0.163 & 0.167 & 0.111 \\
Sample size & 173,554 & 173,554 & 173,554 & 173,554 \\
\midrule 
\midrule 
 \multicolumn{5}{c}{Panel B: Other effects of Medicaid expansion} \\
 Independent Variable & Currently & Employed & Full time & Earned \\
  & married &  &  status & income \\
  & (1) & (2) & (3) &  (4)  \\
\midrule 
 Medicaid expansion*Post   & -0.009   & 0.003   & -0.000   & 140.917  \\
 & (0.009)   & (0.007)   & (0.008)   & (442.269)  \\
 Male   & -0.092***   & 0.239***   & 0.326***   & 10,013.480***  \\
 & (0.005)   & (0.021)   & (0.017)   & (554.391)  \\
 Hispanic origin   & 0.099***   & 0.168***   & 0.138***   & 1,853.155***  \\
 & (0.015)   & (0.010)   & (0.008)   & (425.374)  \\
 Currently married   &   & 0.046***   & 0.062***   & 3,528.414***  \\
 &    & (0.009)   & (0.008)   & (357.107)  \\
\\
Other employment covariates  & X & X & X & X \\
R-2 & 0.060 & 0.098 & 0.136 & 0.091 \\
Sample size & 173,554 & 173,554 & 173,554 & 173,554 \\
\bottomrule
\bottomrule
\end{tabular}
}
\label{tab:cov}
\end{centering}
\begin{spacing}{1}
\begin{footnotesize}

\textit{Notes:} This table explores the effect of the expansion on take up of health insurance by type in Panel A and in other possible outcomes in Panel B. In Panel A, all outcomes are binary indicators indicating belonging to the mentioned group. In Panel B, outcomes from columns 1-3 are binary, while earned income in column 4 is in current US dollars. All regressions include state and year fixed effects with standard errors clustered at the state level, and individual covariates including age, white race, and own educational attainment. The other employment covariates include full time employment status (usual work week at least 40 hours), earned income, and usual worked weekly hours.  Robust standard errors reported in parentheses; these covariates are included if the column shows a checkmark and excluded if it shows an X. Robust standard errors reported in parentheses.  
\end{footnotesize}
\end{spacing}
\end{table}


\newpage
\begin{table}[h!]
\begin{centering}
\singlespacing
\caption{Effects of Medicaid Expansion on Childbearing Outcomes}
\adjustbox{width=\linewidth, center}{
    \begin{tabular}{lcccc}
\toprule
\toprule
 \multicolumn{5}{c}{Panel A: Number of children } \\
 Independent Variable & (1) & (2) & (3) &  (4) \\
\midrule 
 Medicaid expansion*Post   & -0.041***   & -0.097***   & -0.042***   & -0.101***  \\
 & (0.008)   & (0.010)   & (0.008)   & (0.011)  \\
 Has Medicaid coverage   &   & 0.518***   &   & 0.550***  \\
 &    & (0.040)   &    & (0.041)  \\
\\
Employment behavior covariates  &  & X &  & X \\
R-2 & 0.218 & 0.239 & 0.219 & 0.243 \\
Sample size & 173,554 & 173,554 & 173,554 & 173,554 \\
\midrule
\midrule
 \multicolumn{5}{c}{Panel B: Number of children } \\
 Independent Variable & (1) & (2) & (3) &  (4) \\
\midrule 
 Medicaid expansion*Post   & -0.006   & -0.014*   & -0.006   & -0.014*  \\
 & (0.006)   & (0.007)   & (0.006)   & (0.007)  \\
 Has Medicaid coverage   &   & 0.073***   &   & 0.076***  \\
 &    & (0.008)   &    & (0.008)  \\
\\
Employment behavior covariates  &  & X &  & X \\
R-2 & 0.062 & 0.068 & 0.062 & 0.069 \\
Sample size & 173,554 & 173,554 & 173,554 & 173,554 \\
\midrule
\midrule
\bottomrule
\end{tabular}
}
\label{tab:child_did}
\end{centering}
\begin{spacing}{1}
\begin{footnotesize}

\textit{Notes:} This table explores the effect of the expansion on childbearing outcomes. Panel A uses number of children as the outcome variable and Panel B uses having a new baby (a child of age no greater than 1) as the outcome. Columns 2 and 4 in each panel incorporate the effect of having Medicaid and Columns 3 and 4 incorporate employment covariates as detailed in Table \ref{tab:cov}: employed, earned income, weekly work hours, full time status. Robust standard errors shown in parentheses. All regressions include state and year fixed effects, with standard errors clustered at the state level.
\end{footnotesize}
\end{spacing}
\end{table}
\newpage
\pagestyle{empty}
\begin{figure}[ht!]
      \centering
      \caption{Impact on number of children–Event study}
      \label{fig:es_nchild}
      \begin{subfigure}[b]{0.49\textwidth}
        \caption{Overall}
\includegraphics[width=1\linewidth]{output/es_nchild.png}
      \end{subfigure}
       \vspace{.25em} \\
          %%%%%%%%%%%%%% Row 1
            \begin{subfigure}[b]{0.49\textwidth}
            \caption{Male}
\includegraphics[width=1\linewidth]{output/es_nchild_male1.png}
          \end{subfigure}
          \hfill
          \begin{subfigure}[b]{0.49\textwidth}
         \caption{Female}
\includegraphics[width=1\linewidth]{output/es_nchild_male0.png}
          \end{subfigure}
          
          \vspace{.25em}
      
          %%%%%%%%%%%%%% Row 2
          \begin{subfigure}[b]{0.49\textwidth}
            \caption{Hispanic}
\includegraphics[width=1\linewidth]{output/es_nchild_any_hispan1.png}
          \end{subfigure}
          \hfill
          \begin{subfigure}[b]{0.49\textwidth}
          \caption{Non–Hispanic}
\includegraphics[width=1\linewidth]{output/es_nchild_any_hispan0.png}
          \end{subfigure}
      
          \vspace{.25em}
      
          %%%%%%%%%%%%%% Row 3
        \begin{subfigure}[b]{0.49\textwidth}
 \caption{Employed}
\includegraphics[width=1\linewidth]{output/es_nchild_employed1.png}
          \end{subfigure}
          \hfill
          \begin{subfigure}[b]{0.49\textwidth}
 \caption{Not Employed}
\includegraphics[width=1\linewidth]{output/es_nchild_employed0.png}
          \end{subfigure}
\begin{scriptsize}
\textit{Notes:} The event studies explores the effect of the expansion on number of children, using the specification in Column 3 of Table \ref{tab:child_did}. All regressions include state and year fixed effects, with standard errors clustered at the state level. 95\% confidence intervals pictured. Controls include demographics (excluding heterogeneity category) and employment/marriage controls. 
\end{scriptsize}
\end{figure}

% Event study heterogeneity
\begin{figure}[ht!]
      \centering
      \caption{Impact on new baby–Event study}
         \label{fig:es_newbaby}
        \begin{subfigure}[b]{0.49\textwidth}
        \caption{Overall}
\includegraphics[width=1\linewidth]{output/es_newbaby.png}
      \end{subfigure}
       \vspace{.5em} \\
          %%%%%%%%%%%%%% Row 1
            \begin{subfigure}[b]{0.49\textwidth}
            \caption{Male}
\includegraphics[width=1\linewidth]{output/es_newbaby_male1.png}
          \end{subfigure}
          \hfill
          \begin{subfigure}[b]{0.49\textwidth}
         \caption{Female}
\includegraphics[width=1\linewidth]{output/es_newbaby_male0.png}
          \end{subfigure}
          
          \vspace{.5em}
      
          %%%%%%%%%%%%%% Row 2
          \begin{subfigure}[b]{0.49\textwidth}
            \caption{Hispanic}
\includegraphics[width=1\linewidth]{output/es_newbaby_any_hispan1.png}
          \end{subfigure}
          \hfill
          \begin{subfigure}[b]{0.49\textwidth}
          \caption{Non–Hispanic}
\includegraphics[width=1\linewidth]{output/es_newbaby_any_hispan0.png}
          \end{subfigure}
      
          \vspace{.5em}
      
          %%%%%%%%%%%%%% Row 3
        \begin{subfigure}[b]{0.49\textwidth}
 \caption{Employed}
\includegraphics[width=1\linewidth]{output/es_newbaby_employed1.png}
          \end{subfigure}
          \hfill
          \begin{subfigure}[b]{0.49\textwidth}
 \caption{Not employed}
\includegraphics[width=1\linewidth]{output/es_newbaby_employed0.png}
          \end{subfigure}
\begin{scriptsize}
\textit{Notes:} The event studies explores the effect of the expansion on having a new baby, using the specification in Column 3 of Table \ref{tab:child_did}. All regressions include state and year fixed effects, with standard errors clustered at the state level. 95\% confidence intervals pictured. Controls include demographics (excluding heterogeneity category) and employment/marriage controls. 
\end{scriptsize}
\end{figure}

\end{document}
